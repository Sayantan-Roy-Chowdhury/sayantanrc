\documentclass[psamsfonts]{amsart}

%-------Packages---------
\usepackage{amssymb,amsfonts}
\usepackage[all,arc]{xy}
\usepackage{enumerate}
\usepackage{mathrsfs}
\usepackage{tikz-cd}
\usepackage{upgreek}
\usepackage{hyperref}
\usepackage{cleveref}
%--------Theorem Environments--------
%theoremstyle{plain} --- default
\newtheorem{thm}{Theorem}[section]
\newtheorem{cor}[thm]{Corollary}
\newtheorem{prop}[thm]{Proposition}
\newtheorem{lem}[thm]{Lemma}
\newtheorem{conj}[thm]{Conjecture}
\newtheorem{quest}[thm]{Question}

\theoremstyle{definition}
\newtheorem{defn}[thm]{Definition}
\newtheorem{defns}[thm]{Definitions}
\newtheorem{con}[thm]{Construction}
\newtheorem{exmp}[thm]{Example}
\newtheorem{exmps}[thm]{Examples}
\newtheorem{notn}[thm]{Notation}
\newtheorem{notns}[thm]{Notations}
\newtheorem{addm}[thm]{Addendum}
\newtheorem{exer}[thm]{Exercise}

\theoremstyle{remark}
\newtheorem{rem}[thm]{Remark}
\newtheorem{rems}[thm]{Remarks}
\newtheorem{warn}[thm]{Warning}
\newtheorem{sch}[thm]{Scholium}

\newcommand{\vs}{\vspace{0.25mm}}

\makeatletter
\let\c@equation\c@thm
\makeatother
\numberwithin{equation}{section}

\bibliographystyle{plain}

%--------Meta Data: Fill in your info------
\title{Research Diaries 1}

\author{Sayantan Roy Chowdhury}



\begin{document}
\maketitle
\tableofcontents
\section{Motivations}
The motivations are manifold. We want  to capture several facts about curvature in this article. We will start by analysing how  we can make the 'curvature' of plane curves precise.
\subsection{Convention}
To quantize the notion , we need to define a model space where the curvature is +1. By way of convention, we choose this to be $S^{n}$.
\subsection{Curves} 
We parameterize the curve M such that the tangent vectors are of unit length. We define the 'Gauss' map : 
\[   g : M \to S^{1} \]
\[   g(x) = \dot{\gamma}(x) \]
For p $\in$ M , and a ball $B_{r}(p)$ $\subset$ M, we define
\[ \kappa(p) =  \lim_{r \to 0} \dfrac{\text{Volume of } g(B_{r}(p))}{\text{Volume of }B_{r}(p)}  \] 
This captures the infinitesmall change in angle of the tangent vector at point p and hence curvature in keeping with our convention. Whatever
this defination gains in motivation , loses in rigour. However, we can rectify this easily :
\[ \kappa(p) = det(dg_{p}) \]
We leave it for the reader to 'check' that this is a correct mathematical model. And voila, we have recovered
\[\kappa(p) = \| \ddot{\gamma} \| \] 
We invite the reader to check that the curvature of circle of radius r is $\dfrac{1}{r}$ using both the definations.





\subsection{Curves Embedded in a Surface embeddded in $\Re^{3}$ } 
Let $\gamma$ be a curve embedded in a surface M in $\Re^{3}$ . 
We define the geodesic curvature of the curve :
\[ \kappa_{g}(p) = \text{Proj}_{T_{p}(M)}\kappa(p)\]  
Let w be a point particle constrained to move on the surface. Then the path it traces out in the absence of an unbalanced force is the geodesic.
\subsection{Curvature of a surface}
Let M be a  surface embedded in $\Re^{3}$.We attempt to replicate the defination of curvature as in the case of curve:\\
We first define the 'Gauss' map : 
\[   g : M \to S^{2} \]
\[   g(x) = \text{normal vector to M at x}  \]
For p $\in$ M , and a ball $B_{r}(p)$ $\subset$ M, we define
\[ \kappa(p) =  \lim_{r \to 0} \dfrac{\text{Volume of } g(B_{r}(p))}{\text{Volume of }B_{r}(p)}  \] 
And similarly as above 
\[K(p) = \| \det(dg_{p}) \| \]
\subsection{Parallel Transport}
We define parallel transport along a curve $\gamma$ as :
\[ \dot{\theta} = \kappa_{g}(p)                          \]
where $\theta$ is the angle between the vector and $\dot{\gamma}$ 
\paragraph{}
An alternate definition of the parallel transport is : it is a curve 
\[ s: [0,1] \to TM \]
\[ s((t)) \in T_{\gamma(t)}M \]
\[ \text{Proj}_{T_{\gamma(t)}(M)}(\dot{s}) = 0 \]
The two definition are related as follows :
$\theta(t)$ = angle between s(t) and $\dot{\gamma}$ .
\[ \cos(\theta(t)) = \langle s(t), \dot{\gamma}(t)\rangle \text{since these are unit length vectors}                         \]
\[  sin(\theta(t))\dot{\theta}(t) = \langle s(t), \ddot{\gamma}(t)  \rangle \text{since $\dot{s}$ is perpendicular to tangent plane } \]
Now $\kappa_{g}$ is projection of $\ddot{\gamma}$ along the tangent plane whose normal vector is along $\dot{s}(t)$. So
\[ \kappa_{g} = \ddot{\gamma}(t) - \langle \ddot{\gamma}(t) , \dot{s}(t) \rangle \dot{s}(t) \] 
\[ \| \kappa_{g} \| = \|\ddot{\gamma}(t)\|\sin(\phi) \text{where $\phi$ is the angle between $\dot{s}(t)$ and $\ddot{\gamma}(t)$ } 
\]
\[ \langle s(t), \ddot{\gamma}(t) \rangle =  \|\ddot{\gamma}(t)\|\sin(\phi)sin(\theta)                          \]
\section{Gauss map and the holonomy }
We use the "maths" convention of the spherical coordinates . For a sphere this result is quite clear. We can check on a parallel circle at angle $\phi$ the holonomy is $2\pi(1-\sin(\phi))$ . Hence the holonomy form is $(1 - \sin(\phi))d\theta$ . 
\paragraph{}
Now we check that the area form on a sphere is $\cos(\phi)d\theta d\phi$ . Hence we have by stokes theorem :
\[ \int K dA = \text{area of image of gauss map} = \int \kappa_{g} \text{on sphere} = \int \kappa_{g} = \int \dot{\theta} = \theta(1) - \theta(0) \]  


\section{Derivative of tensor}
We first define how to derive a tensor
\[\nabla(T(v_{1},v_{2} .... v_{i},...v_{n})) = \nabla(T)(v_{1},v_{2} .... v_{i},...v_{n}) + T(\nabla v_{1},v_{2} .... v_{i},...v_{n}) + T(v_{1},\nabla v_{2} .... v_{i},...v_{n}) ... + T(v_{1},v_{2} .... \nabla v_{i},...v_{n}) ..... + T(v_{1},v_{2} .... v_{i},...\nabla v_{n}) \]
A similar formula is true for lie derivative as well \\



Let g($\langle , \rangle$) be a Riemannian metric and $\nabla$ be the corresponding Levi Civita Connection. then we have the following interesting formula for the lie derivative :
\[  (L_{X}g)(Y,Z) = \langle \nabla_{Y}X ,Z \rangle  +\langle Y ,\nabla_{Z}X \rangle                                        \]

\paragraph{}
\paragraph{}





\section{Sheaves}


A PreSheaf is a contravariant functor from the category of open sets(with inclusion as morphisms) in a topological space X to the category of some objects \\
Sheafification is the process of identifying the presheaves locally equal to each other and then adding sheaves which locally look like some of the presheaves
\section{Proj and Grass}

The manifold structure of the Projective space $\mathbb{P}^{1}$ is given in the following way: we fix a line x=1 . Let $U_{x}$ be the open set consisting of all lines that is not parallel to this line. Hence we could parameterise this lines with their intersection with the line x=1.Hence this gives us a chart to $\mathbb{C}$.In particular they are parameterised by (1,y) $\to$ y where y $\in$ $\mathbb{C}$ \\
Now we try to give the cell structure to the Grassmannian G(k,n) as well . We can consider the set of all k-planes $\Lambda$ intersecting a fixed n-k plane generically however that wont parameterize the k-plane . We would need k no. of (n-k) planes for that .(since they would intersect in a point and hence determine a line through origin contained in k . We need k such lines to determine a k-plane $\Lambda$. Hence let $H_{1},H_{2} ..... H_{k}$ be k (n-k)-planes. Then we consider an open set U = k-planes $\Lambda$ that intersects this planes generically. Hence this set can be parameterised by this k-intersection points and hence has a bijection to $\mathbb{C}^{k(n-k)}$.\\
Now we can parameterise the k-plane $\Lambda$ by a (k $\times$ n ) matrix whose rows span $\Lambda$. 
 
 \section{Exterior Derivatives}
 We first have the following formula for the exterior derivative of differential forms 
 \[   d\omega(X,Y) = X(\omega(Y)) - Y(\omega(X)) - \omega([X,Y])                               \]
 which can be explained quite clearly using the definition of d$\omega$ as (where $\omega$ is a k-form ):
 \[                   d\omega(v_1,v_2,...v_k,v_{k+1}) = \lim_{t_j \to 0}\dfrac{1}{t_1t_2 \dots t_{k+1}}\int_{bP_t} \omega \]
 where $bP_{t}$ is the oriented boundary of the parallelopiped formed by $v_{i}$ as edges .\\
 Using this we see that d$\omega$(X,Y) at a point is integration of $\omega$ on an infinitesmall parallelopiped formed by X and Y . Now , difference in $\omega$ on the two parallel X edges is given by $Y(\omega(X))$ and similarly $X(\omega(Y))$ on the two parallel Y-edges and the parallelogram might not close up , thats why we need to add $\omega([X,Y])$ term .\\
 Arguing in the same way we have the generalised invariant formula :
 \[      d\omega (V_{0},...,V_{k})=\sum _{i}(-1)^{i}V_{i}\left(\omega \left(V_{0},\ldots ,{\hat {V}}_{i},\ldots ,V_{k}\right)\right)+\sum _{i < j}(-1)^{i+j}\omega \left(\left[V_{i},V_{j}\right],V_{0},\ldots ,{\hat {V}}_{i},\ldots ,{\hat {V}}_{j},\ldots ,V_{k}\right)                       \]  
 \section{Parallel Transport and curvatures}
 The connection here represents the derivative of the parallel transport
 i.e. Let$ P^{t}_{\gamma}$ be the Parallel transport for time t on path $\gamma$. Then we have that :
 \[\nabla_{u}v  = \lim_{t \to 0}\tfrac{P^{-t}_{\gamma}(v(\gamma(t))) -v(\gamma(0))}{t} \] 
 where $\gamma(t)$ is the integral curve of the vector field u. \\
 The Riemannian curvature is generally given by the following formula :
\[       R(u,v)w = \nabla_{v}\nabla_{u}w - \nabla_{u}\nabla_{v}w -\nabla_{[u,v]}w                           \] 
This can be understood the same way as we did with the invariant formulation of the exterior derivative - This gives the limit of the holonomy of w on the boundary of parallelopiped formed by the vectors u and v . \\
Now as explained in the char class document - the holonomy on the boundary is related with the integration of the Gaussian curvature on a surface by some form of a stokes theorem . We will see the relation in form of the sectional curvature . \\
First we note that R(u,v)w  is perpendicular to w(when connection is compatible with the metric) . In particular $\langle R(u,v)w , w \rangle$ = 0 .The proof goes as follows :
\[       2 \langle R(v,u)w , w \rangle = V(U(\langle w , w \rangle )) - U(V(\langle w , w \rangle )) - [U,V](\langle w , w \rangle )  = 0       \] 
In particular the holonomy angle is perpendicular to w as the limit approaches 0. Hence we can also write
\[     R(u,v)w = -KdA(u,v) (J_{0}w)                   \]
where K is the gaussian curvature . and limit of the integration of K over a surface is clearly KdA where dA is the area form .
In particular
\[ \|\langle R(x,y)x,y \rangle\| = KdA(x,y)\|x\|\|y\|\sin(\theta)  \]
where $\theta$ is the angle between x and y .
\[ K = \dfrac{\|\langle R(x,y)x,y \rangle\|}{\|x \times y \|^{2}}\]
We can generalise this notion for higher dimensional manifolds in a
 similar way \\
 Let M be a manifold . x is a point of M . Let u and v be two linearly independent vectors of $T_{x}M$ . We define the sectional curvature 
 \[  K(\sigma) = \dfrac{\|\langle R(u,v)u,v \rangle\|}{\|u \times v \|^{2}}                                              \] 
where $\sigma$ is the plane spanned by u and v . Hence this can be also seen as the sectional curvature of the local surface (formed by the exp map on $T_{x}M$) whose tangent space at point x is $\sigma$ . \\
Now we describe the other two forms of curvature on a manifold . let $\xi$ be an unit vector :
\[ Ric_{p}(\xi , \xi ) =   \dfrac{\sum_{i}\langle R(\xi , e_{i})\xi , e_{i}\rangle}{n-1}                                     \]
where $e_{i}$ is the orthonormal basis of the hyperplane perpendicular to $\xi$ .Hence this is the average of sectional curvatures of all the planes containing $\xi$ . In normal coordiantes, this measures the deviation of the volume form. In particular The normal coordinates is :
\[   g_{ij}=\delta _{ij}+O\left(|x|^{2}\right)   \]
So the metric is given by 
\[    g_{ij}=\delta _{ij}-{\tfrac {1}{3}}R_{ikjl}x^{k}x^{l}+O\left(|x|^{3}\right)               \]
Now sine the volume form is $\sqrt{det (g_{ij})}$ we have :
\[ d\mu _{g}=\left[1-{\tfrac {1}{6}}R_{jk}x^{j}x^{k}+O\left(|x|^{3}\right)\right]d\mu _{\rm {Euclidean}}                 \]
Thus, if the Ricci curvature Ric(ξ,ξ) is positive in the direction of a vector ξ, the conical region in M swept out by a tightly focused family of geodesic segments of length $ \varepsilon$  emanating from p, with initial velocity inside a small cone about ξ, will have smaller volume than the corresponding conical region in Euclidean space, at least provided that $ \varepsilon $  is sufficiently small. Similarly, if the Ricci curvature is negative in the direction of a given vector $\xi$, such a conical region in the manifold will instead have larger volume than it would in Euclidean space. (Read the Global Geometry topology part on wikipedia)\\
We also have this in normal coordinates from the above
\[       R_{ij}=-{\tfrac {2}{3}}\Delta \left(g_{ij}\right)              \] 
Also when a manifold is a hypersurface embedded in an Euclidean space , The principal directions of the hypersurface are the eigendirections of the Ricci tensor.\\ \\
The scalar curvature, in normal coordinates measures the deviation of volume of the geodesic balls .
\[       K(p) = \dfrac{\sum Ric_{p}(e_{i})}{n}                           \]
And in normal coordinates we hae the following 2 expressions :
\[               \frac {\text{Vol} (B_{\varepsilon }(p)\subset M)}{\text{Vol} (B_{\varepsilon }(0)\subset {\mathbb {R} }^{n})}=1-\frac {S}{6(n+2)}\varepsilon ^{2}+O(\varepsilon ^{4}).                    \]
And for the (n-1) balls in their boundary
\[{\frac {\text {Area} (\partial B_{\varepsilon }(p)\subset M)}{\text {Area} (\partial B_{\varepsilon }(0)\subset {\mathbb {R} }^{n})}}=1-{\frac {S}{6n}}\varepsilon ^{2}+O(\varepsilon ^{4})
\]
\section{Connections and Curvatures over vector bundles}

Now we consider a vector bundle E over a manifold M . p $\in$ M and let X be a direction in $T_{p}M$ , s be a section of E. Then we want  to differentiate s in the direction X. The obvious choice is to take the vector $s_{*}(X)$ . However this is an element of TE whereas we want an element of E.  Now the tangent space of a fibre is the vector space itself(Call this bundle VE) . So we want a decomposition 
\[            TE = VE \oplus HE                       \]    
where HE is the horizontal space. In particular we want a splitting of the following exact sequence 
\[      0 \rightarrow ker(\pi_{*}) \rightarrow TE \rightarrow \pi^{*}TM \rightarrow 0                                    \]


 Now we want to define the same concepts of parallel transport in this case. Let $P^{t}_{\gamma}$ be the parallel transport - Then it must independent of the choice of the integral curve and must be linear .
                     


\[       d_{\nabla}^{2} (s) =   F_{\nabla}(s)                           \]
where $F_{\nabla}$ is an End(E) valued 2-form 


There are a multiple ways of defining curvature .A is the connection which gives us the splitting. 
\begin{enumerate}
	\item R(X,Y)v - This is a End(E) valued 2 form on M
	\item  $-A([X_{h},Y_{h}])v$ this is also End(E) valued 2 form on M where $X_{h}$ and $Y_{h}$ are the horizontal lifts . 
	\item $d_{\nabla}A$($X_{h}$,$Y_{h}$) where A is considered as a E-valued 1-form on TE. maybe this is equal to $d{\nabla}$(A) anyway.
	\item dA($X_{h}$,$Y_{h}$)
	\item dA + $\dfrac{A \wedge_{[,]_{E}} A}{2}$ . This gives an E-valued 2 form on TE . Now this is basic i.e. if under the identification TE = TM $\oplus$ E . this is  a map from TM $\to$ (E $\to$ E)
	\item $d_{\nabla}^{2}$(s) where s is a section of E . In particular. if in local coordinates where $\{s_{i}\}$ are the frames : 
	\[     d_{\nabla}(s_{i}) = \sum_{i} \omega_{ij} \otimes s_{j}                                    \]
	Then , we have 
	\[     d_{\nabla}^{2}(s_{i}) =                                   \sum_{j} d\omega_{ij} \otimes s_{j} + \omega_{ij} \otimes d_{\nabla} s_{j} 
	= \sum _{k} (d\omega_{ik} + \sum_{j} \omega_{ij} \wedge \omega_{jk}) \otimes s_{k}         \]
\end{enumerate}





characteristic class as obstruction theory
poincare hopf - lefschetz fixed point theorem = lefchetz no is the intersection number of the graph of the function and diagonal in X $\times$ X = euler characteristic of the fixed point set.\\ 
euler characteristic = self intersection number of a space





\section{Vector bundle theory}
I woul like to document some facts about the vector bundle over here which I would like to use over and over .
\begin{enumerate}
	\item $f_{0}$ $\sim_{homotopy}$ $f_{1}$ $\Rightarrow$ $f_{0}^{*}(E)$ $\cong$ $f_{1}^{*}(E)$
	\item f : X $\to$ Y is a homotopy equivalence $\Rightarrow$ $f^{*}$ : Vect(Y) $\to$ Vect(X) is a bijection
	\item Suppose Y $\subset$ X closed , E is a vector bundle over X . Let $\alpha$ be a trivilization of $E|_{Y}$ . then $E/\alpha$ is a vector bundle over X/Y .
	\item If Y is contractible , then f : X $\to$ X/Y induces a bijection $f^{*}$ : Vect(X/Y) $\to$ Vect(X)
	\item Any exact sequence of vector bundles split
	\item If E is a vector bundle , there exists a vector bundle G such that E $\oplus$ G is a trivial vector bundle.
	\item $Vect_{n}(X)$ $\cong$ $\lim_{m \to \infty}[X,,Gr_{n}(\mathbb{C}^{m})]$ 
	\item Replacing everything in 1 by (G-) works   
\end{enumerate} 
\begin{defn}
	Let A be a semigroup . then A $\xrightarrow{\alpha}$ K(A) is called the Grothendieck group of A if for all homomorphism A $\xrightarrow{b}$ $\exists$ an unique homomorphism K(A) $\xrightarrow{c}$ such that the following diagram commutes 
	\[
	\begin{tikzcd}
	A \arrow{r}{\alpha} \arrow{d}{b} & K(A) \arrow[ld,dashed,"c"] \\
	G
	\end{tikzcd}
	\]
	
\end{defn}
\section{K-theory}
In the next few sections we will see how to construct K(A). For a space X , K(X) denotes the Grothendieck group of the Vect(X) . 
\section{Rrincipal Bundles}
Let $\pi$ : P $\to$ M be a principal bundle .
We have the following G-exact sequence :
\begin{equation}\label{principal connection}
	0 \rightarrow ker(\pi_{*})  \rightarrow TP \xrightarrow{\pi_{*}}  \pi^{*}TM \rightarrow 0                  
\end{equation}
\textbf{$ker(\pi_{*})$} \\                           
$ker(\pi_{*})$ is the vertical part of the bundle (= G) , since it goes to the same point under p.The action of G is just $(m_{g})_{*}$ where $m_{g}$ is just the action of G on P. Also
\[ ker(\pi_{*}) = P \times lie(G) \]
\[ (p,m) \xrightarrow{\psi} \partial_{t}(pe^{tm})\vert_{t=0}\]
Note that 
\[(m_{g})_{*}(\psi(p,m)) = \psi(pg^{-1} , Ad(g)m) \]
where Ad(g) = $(ad(g))_{*}$ on  lie(G) and in particular $gxg^{-1}$ on matrix groups(This is akin to using $gxg^{-1}$ when we change the basis by gx.)
\\
\textbf{$\pi^{*}TM$}  \\
This is the horizontal part of the bundle and the action on it is clear when it is considered as subbundle of P $\times$ TM. 
\\
A connection A on the principal bundle is just a G-equivariant splitting of \cref{principal connection}. It can be treated in the following two ways.
\begin{enumerate}
	\item A G-equivariant map A : TP $\to$ P $\times$ lie(G) such that it is identity o the vertical part.
	\item A identifies $\pi^{*}TM$ as a subbundle of TP : the horizontal bundle $H_{A}$ = ker(A) . G-equivariance implies $(m_{g})_{*}(H_{A})$ = $H_{A}$. Hence for each vector v $\in$ $TM_{x}$, we get $v_{A}$ $\in$ $TP_{(x,g)}$ such that 
	\[ \pi_{*}(v_{A}) = v \]
	i.e. we can horizontally list tangent vectors to TP now
\end{enumerate}
\section{Principal Bundles, Connections and Curvature}
Principal bundles are akin to the notion of symmetry that we use i group theory. On the otherhand ,let $\rho$ : G $\to$ $GL(V)$ be a repn of G in V. we could see P $\times_{\rho}$ V as associating to each point of X a vector space V and the role of G(as a subgroup of GL(V)-the symmetry group of V) is to identify different ways of looking at V .\\
One of the importance of this approach is that G can act as a symmetry group of a lot of vector spaces just by changing the representation.
\[ P \times_{\rho} V = P \times V / \sim \text{ where } (p,v) \sim (pg^{-1},\rho(g)v) \] 
\newpage
\vspace{7mm} 
\begin{center}
	\begin{tabular}{ | p{9 cm} | p{9cm} |}
		\hline\hline
		\vspace{0.25mm}
		Principal bundle P  & \vspace{0.25mm} Associated bundle P $\times_{\rho}V$ \\ [1ex] \hline
		
		
		\vs $s^{P}$ : P $\to$ V , G-equivariant .\newline
		Hence associates to each point a vector(since P $\times_{\rho}$ V is just associating V to each point). The G-equivariance ensures it goes to the same vector under different identifications  & \vs A section s of P $\times_{\rho}$ V \\ [1ex] \hline
		
		
		\vs $s^{P}$ : $\pi^{*}E^{*}$ $\to$ V ,fiberwise linear , G-equivariant  & \vs A section s of P $\times_{\rho}$ V $\otimes$ E where E is a vector bundle. \\ [1ex]
		\hline
		
		\vs connection A : TP $\to$ lie(G) , G-equivariant , also helps to lift vector fields   & \vs covariant derivative $\Delta_{A}$ : $C^{\infty}(E)$ $\to$ $C^{\infty}(E)$ $\otimes$ $T^{*}(M)$ (linear + Leibnitz) .  \newline
		$\Delta_{A}(s)(v)$ =  $s_{*}(v_{A})$                          
		\\ [1ex]
		\hline
	\end{tabular}
\end{center}






\end{document}